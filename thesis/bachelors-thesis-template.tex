\documentclass[12pt]{article}


% A package for setting layout and margins for your thesis 
\usepackage[a4paper]{geometry}


% Use package babel for English or Estonian 
% If you use Estonian make sure that Estonian hyphenation is installed 
% - hypen-estonian or eehyp packages
\usepackage[estonian, english]{babel} 
% \usepackage[english]{babel}
% \usepackage[estonian]{babel}
%
%
% When you write in Estonian then you want to use text with right character set
% By default LaTeX does not know what to do with õäöu letters. You have to specify
% a correct input and font encoding. For that you have to Google the Web     
%
% For TexShop under MacOS X. The right lines are 
%\usepackage[applemac]{inputenc}
\usepackage[T1]{fontenc}
%
% For Windows and Linux the right magic lines are   
% \usepackage[latin1]{inputenc}
% \usepackage[latin5]{inputenc}
% \usepackage[T1]{fontenc}


% General packages for math in general, theorems and symbols 
% Read ftp://ftp.ams.org/ams/doc/amsmath/short-math-guide.pdf for further information
\usepackage{amsmath} 
\usepackage{amsthm}
\usepackage{amssymb}

% Optional calligraphic fonts    
% \usepackage[mathscr]{eucal}

% Packages for building tables and tabulars 
\usepackage{array}
\usepackage{tabu}   % Wide lines in tables
\usepackage{xspace} % Non-eatable spaces in macros

% Including graphical images and setting the figure directory
\usepackage{graphicx}
\graphicspath{{figures/}}

% Packages for getting clickable links in PDF file
\usepackage{hyperref}
\usepackage[all]{hypcap}


% Packages for defining colourful text together with some colours
\usepackage{color}
\usepackage{xcolor} 
%\definecolor{dkgreen}{rgb}{0,0.6,0}
%\definecolor{gray}{rgb}{0.5,0.5,0.5}
\definecolor{mauve}{rgb}{0.58,0,0.82}


% Standard package for drawing algorithms
% Since the thesis in article format we must define \chapter for
% the package algorithm2e (otherwise obscure errors occur) 
\let\chapter\section
\usepackage[ruled, vlined, linesnumbered]{algorithm2e}

% Fix a  set of keywords which you use inside algorithms
\SetKw{True}{true}
\SetKw{False}{false}
\SetKwData{typeInt}{Int}
\SetKwData{typeRat}{Rat}
\SetKwData{Defined}{Defined}
\SetKwFunction{parseStatement}{parseStatement}


% Nice todo notes
\usepackage{todonotes}


% Proper way to create coloured code listings
\usepackage{listings}
\lstset{ 
  %language=python,                % the language of the code
  language=C++,
  basicstyle=\footnotesize,        % the size of the fonts that are used for the code
  %numbers=left,                   % where to put the line-numbers
  %numberstyle=\footnotesize,      % the size of the fonts that are used for the line-numbers
  numberstyle=\tiny\color{gray}, 
  stepnumber=1,                    % the step between two line-numbers. If it's 1, each line 
                                   % will be numbered
  numbersep=5pt,                   % how far the line-numbers are from the code
  backgroundcolor=\color{white},   % choose the background color. You must add \usepackage{color}
  showspaces=false,                % show spaces adding particular underscores
  showstringspaces=false,          % underline spaces within strings
  showtabs=false,                  % show tabs within strings adding particular underscores
  frame = lines,
  %frame=single,                   % adds a frame around the code
  rulecolor=\color{black},		   % if not set, the frame-color may be changed on line-breaks within 
                                   % not-black text (e.g. commens (green here))
  tabsize=2,                       % sets default tabsize to 2 spaces
  captionpos=b,                    % sets the caption-position to bottom
  breaklines=true,                 % sets automatic line breaking
  breakatwhitespace=false,         % sets if automatic breaks should only happen at whitespace
  %title=\lstname,                 % show the filename of files included with \lstinputlisting;
                                   % also try caption instead of title
                                   % also try caption instead of title
  keywordstyle=\color{blue},       % keyword style
  commentstyle=\color{dkgreen},    % comment style
  stringstyle=\color{mauve},       % string literal style
  escapeinside={\%*}{*)},          % if you want to add a comment within your code
  morekeywords={*,game, fun}       % if you want to add more keywords to the set
}


% Obscure packages to write logic formulae and program semantics
% Unless you do a bachelor thesis on program semantics or static code analysis you do not need that
% http://logicmatters.net/resources/ndexamples/proofsty3.html <= writing type rules => use semantic::inference
% ftp://tug.ctan.org/tex-archive/macros/latex/contrib/semantic/semantic.pdf
\usepackage{proof}
\usepackage{semantic} 
\setlength{\inferLineSkip}{4pt}
\def\predicatebegin #1\predicateend{$\Gamma \vdash #1$}

% If you really want to draw figures in LaTeX use packages tikz or pstricks
% However, getting a corresponding illustrations is really painful  


% Define your favorite macros that you use inside the thesis 
% Name followed by non-removable space
\newcommand{\proveit}{ProveIt\xspace}

% Macros that make sure that the math mode is set
\newcommand{\typeF}[1] {\ensuremath{\mathsf{type_{#1}}}\xspace}
\newcommand{\opDiv}{\ensuremath{\backslash \mathsf{div}}\xspace} 

% Nice Todo box
\newcommand{\TODO}{\todo[inline]}

% A way to define theorems and lemmata
\newtheorem{theorem}{Theorem}








%%% BEGIN DOCUMENT
\begin{document}

% BEGIN TITLE PAGE
\thispagestyle{empty}
\begin{center}

\large
UNIVERSITY OF TARTU\\[2mm]
Institute of Computer Science\\
Computer Science Curriculum\\[2mm]

%\vspace*{\stretch{5}}
\vspace{25mm}

\Large Jevgeni Savostkin

\vspace{4mm}

\huge Improving accuracy of BCI with multiple simultaneous users

%\vspace*{\stretch{7}}
\vspace{20mm}

\Large Master's Thesis (30 ECTS)

\end{center}

\vspace{2mm}

\begin{flushright}
 {
 \setlength{\extrarowheight}{5pt}
 \begin{tabular}{r l} 
  \sffamily Supervisor: & \sffamily Ilya Kuzovkin, MSc \\ 
  \sffamily Supervisor: & \sffamily Raul Vicente, PhD 
 \end{tabular}
 }
\end{flushright}

%\vspace*{\stretch{3}}
\vspace{10mm}

%{\noindent Author: .................................................................................... ``.....'' ..........\hskip16pt 2048}
\vspace{2mm}


%{\noindent Supervisor: ............................................................................... ``.....'' ..........\hskip16pt 2048}

\vspace{2mm}

%{\noindent Supervisor: ............................................................................... ``.....'' ..........\hskip16pt 2048}

\vspace{8mm}


\vfill
\centerline{Tartu 2017}

% END TITLE PAGE

% If the thesis is printed on both sides of the page then 
% the second page must be must be empty. Comment this out
% if you print only to one side of the page comment this out
% \newpage
% \thispagestyle{empty}    
% \phantom{Text to fill the page}
% END OF EXTRA PAGE WITHOUT NUMBER

\newpage
\selectlanguage{english}
\noindent\textbf{\large Improving accuracy of BCI with multiple simultaneous users}
\vspace*{2ex}
{\flushleft{\textbf{Abstract:}} }
Many interpreting program languages are dynamically typed, such as Visual Basic or Python. As a result, it is easy to write programs that crash due to mismatches of provided and expected data types.  One possible solution to this problem is automatic type derivation during compilation. In this work, we consider study how to detect type errors in the \textsc{Whitespace} language by using fourth order logic formulae as annotations. The main result of this thesis is a new triple-exponential type inference algorithm for the fourth order logic formulae. This is a significant advancement as the question whether there exists such an algorithm was an open question. 
All previous attempts to solve the problem lead lead to logical inconsistencies or required tedious user interaction in terms of interpretative dance. Although the resulting algorithm is slightly inefficient, it can be used to detect obscure programming bugs in the \textsc{Whitespace} language. The latter significantly improves productivity. Our practical experiments showed that productivity is comparable to average Java programmer.   
From a theoretical viewpoint, the result is only a small advancement in rigorous treatment of higher order logic formulae. The results obtained by us do not generalise to formulae with the fifth or higher order. 


\vspace*{3ex}
{\flushleft{\textbf{Keywords:}\TODO{List of keywords}}}
\vspace*{3ex}

\noindent\textbf{CERCS:}\TODO{CERCS code and name:~\url{https://www.etis.ee/Portal/Classifiers/Details/d3717f7b-bec8-4cd9-8ea4-c89cd56ca46e}}

\selectlanguage{english}


\newpage
\tableofcontents



\newpage
\section{Introduction}

Tip: if it's hard for you to start writing, then try to split it to smaller parts, e.g. if the title is ``Type Inference for a Cryptographic Protocol Prover Tool'' then the ``What is it'' can be divided into ``what is type inference'', ``what is cryptographic protocol'' and ``what is the prover tool''. These three can also be split to smaller parts etc.








\newpage
\section{Related work review} 
\TODO{Short description of what this section is about}


\subsection{Title of Subsection 1}

\paragraph{Cross-references to figures, tables and other document elements.}
LaTeX  internally numbers all kind of objects that have sequence numbers:
\begin{itemize}
\item chapters, sections, subsections;
\item figures, tables, algorithms;
\item equations, equation arrays.
\end{itemize}
To reference them automatically, you have to generate a label using \texttt{$\backslash$label\{some-name\}} just after the object that has the number inside. Usually, labels of different objects are split into different namespaces by adding dedicated prefix, such as \texttt{sec:}, \texttt{fig:}. To use the corresponding reference, you must use command \texttt{$\backslash$ref} or \texttt{$\backslash$eqref}. For instance, we can reference this subsection by calling Section~\ref{sec:using_ref}. Note that there should be a nonbreakable space \texttt{\~} between the name of the object and the reference so that they would not appear on different lines.          



\paragraph{Citations.}
Usually, you also want to reference articles, webpages, tools or programs or books. For that you should use citations and references. The system is similar to the cross-referencing system in LaTeX. For each reference you must assign a unique label. Again, there are many naming schemes for labels. However, as you have a short document anything works. To reference to a particular source you must use \texttt{$\backslash$cite\{label\}} or \texttt{$\backslash$cite[page]\{label\}}. 

References themselves can be part of a LaTeX source file. For that you need to define a bibliography section. However, this approach is really uncommon. It is much more easier to use BibTeX to synthesise the right reference form for you. For that you must use two commands in the LaTeX source
\begin{itemize}
\item $\backslash$bibliographystyle\{alpha\} or $\backslash$bibliographystyle\{plain\}
\item $\backslash$bibliography\{file-name\}
\end{itemize}
The first command determines whether the references are numbered by letter-number combinations or by cryptic numbers. It is more common to use \texttt{alpha} style. The second command determines the file containing the bibliographic entries. The file should end with \texttt{bib} extension. Each reference there is in specific form. The simplest way to avoid all technicalities is to use graphical frontend  Jabref (\url{http://jabref.sourceforge.net/}) to manage references. Another alternative is to use DBLP database of references and copy BibTeX entries directly form there.   
    
   
The following paragraph shows how references can be used. Game-based proving is a way to analyse security of a cryptographic protocol~\cite{GameB_1, GameB_2}. There are automatic provers, such as {CertiCrypt\-}~\cite{certicrypt} and ProVerif~\cite{proVerif}.




\newpage
\section{Technical background}


Here are a few examples of how to add figures or pictures to your thesis (see Figures~\ref{fig:fnCompModel}, \ref{fig:game-based_proofs}, \ref{fig:proveit_screenshot}).

Rule: All the figures, tables and extras in the thesis have to be referred to somewhere in the text.

Tip: If you add a screenshot then labeling the parts might help make the text more understandable (panel C vs bottom left part), e.g.
 
Example: A screenshot of \proveit can be seen on Figure~\ref{fig:proveit_screenshot}. The user first enters the pseudocode of the initial game in panel B. \proveit also keeps track of all the previous games showing the progress on a graph seen in panel A.






\clearpage %if newpage doesn't work
\section{Other Ways to Represent Data}

\subsection{Tables}

\begin{table}[h]
\centering
\begin{tabular}{| l | l |}
	\hline
	\bf{Statement} & \bf{Typeset Example} \\
	\hline
	assignment & $a := 5 + b$ \\
	\hline
	uniform choice & $m <- M$ \\
	\hline
	function signature & $f : K \times M -> L$\\
	\hline
\end{tabular}
\caption{Statements in the \proveit language}
\label{tab:statements}
\end{table}


\subsection{Lists}

Numbered list example:
\begin{enumerate}
	\item item one; 
	\item item two;
	\item item three.
\end{enumerate} 

\subsection{Math mode}
Example:
\[
a + b = c + d
\]
Aligning:
\begin{align*}
	a &= 5 \\
	b + c &= a \\
	a -2*3 &= 5/4
\end{align*}
Hint: Variables or equations in text are separated with \$ sign, e.g. $a$, $x - y$.

\paragraph{Inference Rules}
\[ 
	\inference[addition]{x : T & y : T}{x + y : T} 
\]
Bigger example:
\[
\inference[assign]{c := a + b & 
	\inference[addG]{a : \typeRat & 
		\inference[var]{b : \typeInt & \typeInt \subseteq \typeRat}{b : \typeRat}
		}{a + b : \typeRat}
	}{c : \typeRat}
\]


\subsection{algorithm2e}

\begin{algorithm} [!h]
	\caption{typeChecking} \label{alg:typeChecking}
	\KwIn{Abstract syntax tree}
	\KwResult{Type checking result; In addition, type table \typeF{type\_G} for global variables, \typeF{game} for the main game and \typeF{fun} for each $fun \in F$}
	\SetKwData{s}{s}
	\BlankLine
	
	\While{something changed in last cycle}{
		\lForEach{global statement \s} {
			\parseStatement{\s, \typeF{type\_G}}\;
		}
		\ForEach{function $fun$} {
		\lForEach{statement \s in $fun$} {
			\parseStatement{\s, \typeF{fun}}\;
		}
		}
		\lForEach{statement \s in game} {
			\parseStatement{\s, \typeF{game}}\;
		}
	}
	%\eIf{error messages were found}{\Return \False\;}{\Return \True\;}
\end{algorithm}

\subsection{Pseudocode}

\begin{figure} [htb]
\begin{lstlisting}
expression
  : NUMBER
  | VARIABLE
  | '+' expression
  | expression '+' expression
  | expression '*' expression
  | function_name '(' parameters ')'
  | '(' expression ')'
\end{lstlisting}
\caption{Grammar of arithmetic expressions}
\label{fig:parser_exp}
\end{figure}

\subsection{Frame Around Information}

Tip: We can use minipage to create a frame around some important information.
\begin{figure} [h]
\frame{
\begin{minipage}{\textwidth}
\begin{enumerate}
	\item integer division ($\opDiv$) - only usable between \typeInt types
	\item remainder ($\%$) - only usable between \typeInt types
\end{enumerate}
\end{minipage}
}
\caption{Arithmetic operations in \proveit revisited}
\label{fig:aritmOp_revisit}
\end{figure}



\clearpage
\section{Conclusion} 

\TODO{what did you do?} 
\TODO{What are the results?}
\TODO{future work?}


\newpage

\section{Eestikeelne pealkiri}
Bakalaureusetöö(6 EAP) \\
Eesnimi Perekonnanimi \\
Resümee \\


\TODO{Use introduction and conclusion to give a brief overview of what this thesis is about}

\selectlanguage{english}

\newpage
\bibliographystyle{alpha}
\bibliography{bachelor-thesis}

\appendix
\pagebreak
\section*{\small Non-exclusive licence to reproduce thesis and make thesis public}


I, Alice Cooper (date of birth: 4th of February 2048),

\begin{tabbing}
\= Xiii\=\kill
\>1. \> herewith grant the University of Tartu a free permit (non-exclusive licence) to:\\\\ 

\>1.1\> 
\begin{minipage}[t]{14.2cm}
reproduce, for the purpose of preservation and making available to the public, including for addition to the DSpace digital archives until expiry of the term of validity of the copyright, and
\end{minipage}
\\\\
\>1.2 
\begin{minipage}[t]{14.2cm}
make available to the public via the web environment of the University of Tartu, including via the DSpace digital archives until expiry of the term of validity of the copyright,\\ 

Type Inference for a Fourth Order Logic Formulae\\   

supervised by Axel Rose and May Flower

\end{minipage}\\\\ 
\>2. \>I am aware of the fact that the author retains these rights.\\\\
\>3. \>
\begin{minipage}[t]{14.2cm}
I certify that granting the non-exclusive licence does not infringe the intellectual property rights or rights arising from the Personal Data Protection Act. 
\end{minipage}\\
\end{tabbing}

\noindent
Tartu/Tallinn/Narva/Pärnu/Viljandi, dd.mm.yyyy


\end{document}
